\documentclass[10pt]{beamer}

\usepackage{packages}
\title{Exercício Programa 3}
\subtitle{Simulador de sistema de arquivos}
\institute{IME-USP}
\author{Lucas Paiolla Forastiere, 11221911\\ Marcos Siolin Martins, 11221709}
\date{07 de dezembro de 2020}

\begin{document}
    \maketitle
    \section{Sobre o simulador}
    \begin{frame}{Detalhes de Implementação - o arquivo}
        \begin{itemize}
            \justifying
            \item A representação do sistema de arquivos é armazenada em um arquivo que sempre ocupa \texttt{100MB} no sistema de arquivos real. Caso seja executado \texttt{mount} sobre um arquivo que não exista, será gerado um novo arquivo com \texttt{100MB} onde estará armazenado o \texttt{Bitmap}, a \textt{FAT} e o diretório \texttt{root} (/);
            \item Consideramos que conteúdo de arquivos contém apenas caracteres que ocupam 1 \texttt{byte} em seus nomes e conteúdos, ou seja, que pertencem à tabela \texttt{ASCII}. Isso nos permite controlar quanto espaço cada arquivo ou diretório ocupa;
        \end{itemize}
    \end{frame}
    \begin{frame}{Detalhes de Implementação - o arquivo}
        \begin{itemize}
            \justifying
            \item Utilizamos o caractere \texttt{|} (pipe) como separador para indicar situações como o fim de nome de arquivo ou fim de conteúdo, então é importante que não existam arquivos que contenham esse caractere no nome ou em seu conteúdo;
            \item Para preencher espaços em branco utilizamos a constante \texttt{CHAR\_NULO} que é um \texttt{' '} (whitespace).
            \item Após dar mount no arquivo que guarda o sistema de arquivos simulado, o conteúdo do arquivo é trazido para memória e as alterações são feitas em memória. As alterações serão gravados no arquivo em disco quando o comando umount for dado.
        \end{itemize}
    \end{frame}
    \begin{frame}{Detalhes de Implementação - o bitmap}
        \begin{itemize}
            \justifying
            \item O Bitmap é implementado como um vetor booleano de tamanho \texttt{NUM\_BLOCOS}, que é a constante que guarda a quantidade de blocos disponíveis para o sistema de arquivos simulado, desconsiderando os blocos necessários para armazenar o Bitmap e a FAT. O valor \texttt{1/true} indica que o bloco está livre e o valor \texttt{0/false} indica que o bloco está ocupado.
            \item O Bitmap ocupa os primeiros 7 blocos do sistema de arquivos simulado, pois precisa armazenar \texttt{NUM\_BLOCOS bytes}. O espaço restante no 7º bloco é desperdiçado;
        \end{itemize}
    \end{frame}
    \begin{frame}{Detalhes de Implementação - a FAT}
        \begin{itemize}
            \justifying
            \item A FAT é implementada como um vetor de inteiros de tamanho \texttt{NUM\_BLOCOS}. O valor em \texttt{ponteiro[i]} indica qual é o próximo bloco após o $i$ na lista ligada do arquivo. Caso esse valor seja igual à \texttt{BLOCO\_NULO} (um valor de um bloco que não existe), então o bloco $i$ é o último na sequência da lista ligada.
            \item Para armazenar esses ponteiros os convertemos para uma string com tamanho fixo $5$, assim, se \texttt{ponteiro[i] = 1}, no sistema de arquivos simulado será armazenado como \texttt{00001}.
            \item A FAT é armazenada nos 32 blocos conseguintes ao Bitmap, pois precisa armazenar \texttt{NUM\_BLOCOS*5 bytes}. O espaço restante no 32º bloco é desperdiçado;
        \end{itemize}
    \end{frame}
    \begin{frame}{Detalhes de Implementação - o root}
        \begin{itemize}
            \justifying
            \item O diretório \texttt{/} é um diretório especial. Ele está sempre ocupando o bloco 0 (a partir de agora desconsideraremos os blocos necessário para armazenar o bitmap e a FAT) e também armazena os próprios metadados, nessa ordem:
            \begin{itemize}
                \justifying
                \item Tempo Criado - ocupa 10 bytes. É a quantidade em segundos devolvida por \texttt{time(NULL)} no momento de criação do arquivo;
                \item Tempo Modificado - ocupa 10 bytes. É a quantidade em segundos devolvida por \texttt{time(NULL)} no momento de última modificação do arquivo;
                \item Tempo Acesso - ocupa 10 bytes. É a quantidade em segundos devolvida por \texttt{time(NULL)} no momento de último acesso do arquivo;
                \item Nome - ocupa x bytes. Ao fim do nome estará o caractere \texttt{'|'}.
            \end{itemize}
        \end{itemize}
    \end{frame}
    \begin{frame}{Detalhes de Implementação - os diretórios}
        \begin{itemize}
            \justifying
            \item Os diretórios armazenam os metadados dos subdiretórios e dos arquivos que estão ``imediatamente abaixo'' do diretório. Os metadados são armazenados na seguinte ordem:
            \begin{itemize}
                \justifying
                \item Ponteiro para o nome - ocupa 8 bytes. Aponta para o endereço do disco onde está o nome do arquivo/diretório;
                \item Diretório - ocupa 1 byte. Indica se os metadados são de um diretório ou de um arquivo;
                \item Número do primeiro bloco - ocupa 5 bytes. Aponta para o primeiro bloco onde o arquivo/diretório está armazenado; 
                \item Tempo Criado - ocupa 10 bytes;
                \item Tempo Modificado - ocupa 10 bytes;
                \item Tempo Acesso - ocupa 10 bytes;                
            \end{itemize}
        \end{itemize}
        
    \end{frame}
    \begin{frame}{Detalhes de Implementação - os diretórios}
        \begin{itemize}
            \justifying
            \begin{itemize}
                \justifying
                \item Tamanho - ocupa 8 bytes. Se os metadados são de um diretório, então esse valor é sempre 0;
                \item Nome* - ocupa x bytes. Ao fim do nome estará o caractere \texttt{'|'}.
            \end{itemize}
            \item Os diretórios seguem a estratégia apresentada em aula onde cada subarquivo/subdiretório tem um campo de metadados, com exceção do nome (*), de tamanho fixo e o metadado nome fica armazenado em uma região especial chamada de heap. No campo de metadados existe um ponteiro para o lugar onde o nome está armazenado;
            \item Na nossa implementação a heap está armazenada imediatamente após acabarem todos os campos para os metadados. O começo da heap é indicado pelo caractere \texttt{'|'}.
        \end{itemize}
    \end{frame}
\end{document}
